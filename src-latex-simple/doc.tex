\documentclass[preprint,10pt]{sigplanconf}
% The following \documentclass options may be useful:
%
% 10pt          To set in 10-point type instead of 9-point.
% 11pt          To set in 11-point type instead of 9-point.
% authoryear    To obtain author/year citation style instead of numeric.

\usepackage{graphicx}
\usepackage{amsmath}
\usepackage{amssymb}
% \usepackage{amsthm}
\usepackage{array}
% \usepackage{cite}
\usepackage{hyperref}
\usepackage{listings}
\usepackage[nooneline,tight]{subfigure}
\usepackage[usenames,dvipsnames]{xcolor}
\usepackage{tikz}
\usepackage{booktabs}
\usepackage{microtype}
\usepackage{multirow}
\usepackage{flushend}
\usepackage{stmaryrd}
\usepackage[T1]{fontenc}
\usepackage{xspace}
\usepackage{xparse}
\usepackage{subdepth}
\usepackage{wrapfig}
\usepackage[inference]{semantic}
\usepackage{scalerel}
\usepackage{oz}

\usetikzlibrary{positioning,chains,shapes.arrows,shapes.geometric,fit,calc,arrows,decorations.pathmorphing}

\newif\iftechreport
%\techreporttrue % uncomment to activate tech-report setup
\techreportfalse % uncomment to deactivate tech-report setup

\newcommand{\proofsketch}[1]{{\def\proofname{Proof sketch}\proof{#1}}}

% this are the defaults from article.cls
% \renewcommand\floatpagefraction{.9}
% \renewcommand{\topfraction}{.9}
% \renewcommand{\bottomfraction}{.1}
% \renewcommand\textfraction{.1}  

% \addtolength{\abovecaptionskip}{-0.4cm}
% \newcommand{\captionskip}{1.5ex}
% \addtolength{\belowcaptionskip}{-\captionskip}

\hyphenation{meta-lan-guage}
\hyphenation{meta-lan-guages}
\hyphenation{SugarJ}
\hyphenation{JastAddJ}
\hyphenation{name-space}

\begin{document}

\title{A Sound and Optimal Incremental Build System}
%\subtitle{Consistent, Incremental, Self-Applicable Building}
% alternative titles: 
%
% A Modular Build System with Customizable File Stamps
%
% A Modular Build System with Semantic File Stamps
%

\authorinfo{%
  Sebastian Erdweg \and
  Moritz Lichter \and
  Manuel Weiel %
}
{TU Darmstadt, Germany}

\maketitle

\newcommand{\projectname}{ClearDep\xspace}

\begin{abstract}
  Build systems are used in all but the smallest software projects to invoke the
  right build tools on the right files in the right order.  A build system must
  be sound (after a build, generated files consistently reflect the latest
  source files) and efficient (recheck and rebuild as few build units as
  possible). Contemporary build systems provide limited efficiency because they
  lack support for expressing fine-grained file dependencies.

  We present a modular build system called \projectname that supports the
  definition of reusable, parameterized, interconnected builders. When run, a
  builder notifies the build system about required and produced files as well as
  about other builders whose results are needed. To support fine-grained file
  dependencies, we generalize the traditional notion of time stamps to allow
  builders to declare their actual requirements on a file's content.
  \projectname collects the requirements and products of a builder with their
  stamps in a build summary. This enables \projectname to provide provably
  sound and optimal incremental rebuilding, even after changing a builder
  definition itself. We have developed \projectname as a Java API and used it to
  implement numerous builders. We describe our experience with migrating a
  larger ANT build script to \projectname and compare the respective
  build times.
\end{abstract}

\section{Introduction}
This is an introduction text. It can cite references~\cite{Erdweg12thesis}.


\bibliographystyle{abbrv}
\bibliography{bib}


\end{document}


%%% Local Variables: 
%%% mode: latex
%%% TeX-master: t
%%% End: 
