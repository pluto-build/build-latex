\subsection{Subtyping}
\label{sec:subtyping}

As final extension, we consider a co-contextual formulation of PCF with
subtyping. Subtyping is an interesting extension because it affects the
semantics of a typing contexts and, hence, context requirements. In particular,
subtyping weakens the assumptions about variable bindings $x\ofType T$ in typing
contexts. In standard PCF, $(\ctxplus{x\ofType T_x}{\Gamma})$ means that
variable $x$ has type $T_x$ exactly. In contrast, in PCF with subtyping,
$(\ctxplus{x\ofType T_x}{\Gamma})$ means that $T_x$ is an upper bound of the
type of values substitutable for $x$: All values must at least adhere to
$T_x$. Dually, a type annotation $T_x$ on a $\lambda$-abstraction is a lower
bound for the type required for $x$ in subexpressesions: Subexpressions can at
most require $T_x$. Thus, subtyping affects the merging and satisfaction of
context requirements.

We adapt the definition of \reqmergeF to correctly combine requirements from
different subexpressions. Due to subtyping, different subexpressions can require
different types on the same variable. As consequence, a variable has to
simultaneously satisfy the requirements of all subexpressions that refer to
it. That is, when we merge overlapping context requirements, we require the type
of shared variables to be a subtype of both originally required types:
% For example, suppose type checking two separate branches of an expression yields the requirements
% $R_{1} = \{x\ofType U_{1} \}$ and $R_{2} = \{x\ofType U_{2} \}$, i.e.,
% $x$ is required to have at most type $U_{1}$ in one branch and at most type $U_{2}$ in the other.
% A merge should preserve both requirements, which translates to taking the greatest lower bound
% of the two types. That is, the merging of $R_{1}$ and $R_{2}$
% yields $R = \{x \ofType U_{3} \}$ and the constraint $C = \{ U_{3} = U_{1} \wedge U_{2}\}$
% for a fresh type variable $U_{3}$. That is, merging yields a more precise upper bound approximation.
% In summary, the definition of the merge function changes as follows:\TODO{oliver}{not sure if we should add colors to the definition}

\vspace{1ex}
$
\begin{array}{llll}
  \multicolumn{4}{l}{  \reqmerge{\reqs_1}{\reqs_2} = \reqs|_C} \\
  \textit{where}\ & X & = & \dom{\reqs_{1}} \cap \dom{\reqs_{2}} \\
                  & U_x & = & \textit{fresh variable for each } {x \in  X } \\ 
                  & \reqs & = & (\reqs_1 - \dom{\reqs_2}) \cup (\reqs_2 - \dom{\reqs_1})  \cup \{ x\ofType U_x \WHERE x \in X  \} \\
                  & C & = & \{U_x \subtpe \reqs_1(x), U_x \subtpe \reqs_2(x) \WHERE x \in X\}
\end{array}
$
\vspace{1ex}


% Note that for simplicity we only consider a subtype relation 
% with a greatest $\top$ type, but no least $\bot$ type \todo{seba}{necessary to mention this (and so early)?}, similarly to \cite{pierce2002types}.
% It is straightforward to extend our co-contextual formulation with a $\bot$
% type and obtain a complete lattice for types \todo{seba}{is it really straightforward?}.

% Let us first recapitulate the changes to standard PCF for subtyping 
% determine the impact of these changes on the co-contextual type system.
% The rule for applications adds a subtype check, which permits more specific types as function
% arguments than required by the function type and is defined as follows:
% %
% \[
% \inference[\rulename{T-App-Alg}]{\judge{\Gamma}{e_{1}}{T_{1}\to T_{2}} & \judge{\Gamma}{e_{2}}{T_{11}} & \typecolor{T_{11}}\subtpe \typecolor{T_{1}}}{\judge{\Gamma}{\app{e_{1}}{e_{2}}}{T_{2}}.}
% \]
% %
% Similarly, the rule for conditionals is more permissive by computing the least upper
% bound (or join) of the branch types:
% %
% \[
% \inference[\rulename{T-If0-Alg}]{\judge{\Gamma}{e_{1}}{\TNum} & \judge{\Gamma}{e_{2}}{T_{2}} & \judge{\Gamma}{e_{3}}{T_{3}} \\ \typecolor{T} = \typecolor{T_{2} \vee T_{3}}}{\judge{\Gamma}{\ifzero{e_{1}}{e_{2}}{e_{3}}}{T}}
% \]
%

\noindent We do not stipulate a specific subtyping relation. However, we add new
forms of type constraints with standard semantics to express subtyping, joins,
and meets:
% We assume the existence of a constraint solver supporting these constraints:

\vspace{1ex}
$
\newcommand{\comment}[1]{\hskip2em\text{#1}}
\begin{array}{c@{\hskip.5em}c@{\hskip.5em}lll}
  c & ::= & \ldots 
     &\OR  \conssub{T}{T} & \comment{subtype constraint} \\
    & & &\OR  \consjoin{T}{T}{T} & \comment{least upper bound constraint (join)} \\
    & & &\OR  \consmeet{T}{T}{T} & \comment{greatest lower bound constraint (meet)}
\end{array}
$
\vspace{1ex}
 
\noindent A least upper bound constraint $\consjoin{T_{1}}{T_{2}}{T_{3}}$ states
that type $T_{1}$ is the least type in the subtype relation such that both
$T_{2}$ and $T_{3}$ are subtypes of $T_{1}$. A greatest lower bound constraint
$\consmeet{T_{1}}{T_{2}}{T_{3}}$ states that the type $T_{1}$ is the greatest
type in the subtype relation such that both $T_{2}$ and $T_{3}$ are supertypes
of $T_{1}$.

% A property of \pcfsub{} is that for a given expression $e$, a successful type check always
% yields the unique minimal type for $e$ in the subtype relation~\cite{pierce2002types}.
% Accordingly, our co-contextual rules guarantee the same property.

\begin{figure}[t]
  \centering
  {
  \begin{gather*}      
  \inference[\rulename{T-Abs}]
        {\cojudge {e} {T_2} {C} {\reqs} &
         \typectxcolor{C_x} = \{\typecolor{T_1 \subtpe \ctxcolor{\reqs(x)}} \OR \textit{if}\ \ctxcolor{x \in \dom{\reqs}}\}}
        {\cojudge {\abs{x}{T_1}{e}} {T_1 \to T_2} {C \cup \typectxcolor{C_x}} {\reqs - x}}
\\[2ex]
      \inference[\rulename{T-App}]
        {\cojudge {e_1} {T_1} {C_1} {\reqs_1} &
         \cojudge {e_2} {T_2} {C_2} {\reqs_2} \\
         \fresh{U_{1}, U_{2}} & \ctxcolor{\reqmerge{\reqs_1}{\reqs_2} = \reqs|_{\typectxcolor{C}}}}
        {\cojudge {\app{e_1}{e_2}} {U_{2}} {C_1 \cup C_2 \cup \{T_1 = U_{1} \to U_{2}, \conssub{T_{2}}{U_{1}}\} \cup \typectxcolor{C}} {\reqs}}
\\[2ex]
      \inference[\rulename{T-If0}]
        {\cojudge {e_1} {T_1} {C_1} {\reqs_1} &
         \cojudge {e_2} {T_2} {C_2} {\reqs_2} &
         \cojudge {e_3} {T_3} {C_3} {\reqs_3} \\
         \ctxcolor{\reqmerge{\reqs_2}{\reqs_3} = \reqs_{2,3}|_{\typectxcolor{C_{2,3}}}} &
         \ctxcolor{\reqmerge{\reqs_1}{\reqs_{2,3}} = \reqs_{1,2,3}|_{\typectxcolor{C_{1,2,3}}}} &
         \fresh{U} \\
         \typecolor{C =  \{T_1 = \TNum, \consjoin{U}{T_{2}}{T_{3}} \} \cup C_1 \cup C_2 \cup C_3 \cup \typectxcolor{C_{2,3}} \cup \typectxcolor{C_{1,2,3}} }
         }
        {\cojudge {\ifzero{e_1}{e_2}{e_3}} {T_2} {C } {\reqs_{1,2,3}} }
\\[2ex]
  \inference[\rulename{T-Fix}]
        {\cojudge {e} {T} {C} {\reqs} &
         \fresh{U_{1}, U_{2}}}
        {\cojudge {\fix{e}} {U_{1}} {C \cup \{T = U_{1} \to U_{2}, \conssub{U_{2}}{U_{1}}\}} {\reqs}}
\end{gather*}
}
  \caption{Co-contextual type rules for PCF with subtyping.}
  \label{fig:co-pcf-sub}
\end{figure}

%%% Local Variables:
%%% mode: latex
%%% TeX-master: "document"
%%% End:


Figure~\ref{fig:co-pcf-sub} shows the co-contextual rules for PCF enriched with
subtyping. Only the rules for $\lambda$-abstractions, applications,
conditionals, and fixpoints change with respect to the co-contextual PCF type system
in Figure~\ref{fig:co-pcf}. First, consider the rule for $\lambda$-abstraction
\rulename{T-Abs}. As discussed above, a context requirement on $x$ only
describes an upper bound on the declared type of $x$. Or conversely, the
declared type of a variable $x$ is a lower bound on what subexpressions can
require for $x$. Accordingly, we replace the type-equality constraint by a
subtype constraint $\typecolor{T_1 \subtpe \ctxcolor{R(x)}}$.

The other rules \rulename{T-App}, \rulename{T-If0}, and \rulename{T-Fix} are
straightforward extensions to allow for subtyping. In rule \rulename{T-App}, we
allow the argument to be of a subtype of the function's parameter type as
usual. Rule \rulename{T-If0} declares the type of the conditional to be the
least upper bound of the two branch types, which we express by the least upper
bound constraint $\consjoin{U}{T_{1}}{T_{2}}$. In rule \rulename{T-Fix}, we
permit the fixed function to produce values whose type is a subtype of the
function's parameter type.

To illustrate co-contextual type checking with subtypes, consider PCF with
records and the usual depth and with subtyping for records. When type checking
the expression $e = \add{x.m}{x.n}$ with free $x$, we get the following
derivation:

\begin{center}
\begin{tikzpicture}[exp tree]
  \node (add) at (0,4) {\addNode};
  \node (m) at (-2.5,2.5) {$.m$};
  \node (x1) at (-2.5,1.0) {$x$};
  \node (n) at (2.5,2.5) {$.n$};
  \node (x2) at (2.5,1.0) {$x$};

  % \node[type node, right=of lambda-f] (lambda-f-type) {\typeNode{(\alpha\to\beta) \to \alpha \to \beta}};
  % \node[ctx node, right=of lambda-f-type]  (lambda-f-ctx)  {\coctxNode{\emptyset}};
  % \node[type node, right=of lambda-x] (lambda-x-type) {\typeNode{\alpha \to U_2}};
  % \node[ctx node, right=of lambda-x-type]  (lambda-x-ctx)  {\coctxNode{f \ofType \alpha \to U_2}};
  % \node[type node,right=of app] (app-type) {\typeNode{U_2}};
  % \node[ctx node, right=of app-type]  (app-ctx)  {\coctxNode{\coctxplus{f \ofType U_1 \to U_2}{x \ofType U_1}}};
  % \node[type node,right=of f] (f-type) {\typeNode{U_0}};
  % \node[ctx node, right=of f-type]  (f-ctx)  {\coctxNode{f \ofType U_0}};
  \annotateNode{x1}{U_1}{\emptyset}{x \ofType U_1}
  \annotateNode{x2}{U_2}{\emptyset}{x \ofType U_2}
  \annotateNode{m}{U_3}{\{U_3 = U_1.m\}}{x \ofType U_1}
  \annotateNode{n}{U_4}{\{U_4 = U_2.n\}}{x \ofType U_2}
  \annotateNode{add}{\TNum}{%
    \begin{array}{lll}
      \{ & U_3 = U_1.m, U_4 = U_2.n & \\
         & U_3 = \TNum, U_4 = \TNum & \\
         & U_5 \subtpe U_1, U_5 \subtpe U_2 & \} \\
    \end{array}
}{x \ofType U_5}

 \path (add) edge (m);
 \path (add) edge (n);
 \path (m) edge (x1);
 \path (n) edge (x2);


\end{tikzpicture}
\end{center}

\noindent We can simplify the resulting constraint set by eliminating $U_3$ and
$U_4$ to get $\{\TNum = U_1.m, \TNum = U_2.n, U_5 \subtpe U_1, U_5 \subtpe
U_2\}$, where $U_5$ is the type required for $x$. Importantly, the type of $x$
must be a subtype of $U_1$ and $U_2$, which in turn must provide fields $m$ and
$n$ respectively. Thus, the type $U_5$ of $x$ must be a record type that at
least provides fields $m$ and $n$. Indeed, when we close above expression $e$ as
in $\abs{x}{T}{e}$, type rule \rulename{T-Abs} yields another constraint $T
\subtpe U_5$. Accordingly, type checking succeeds for an annotation $T=\{m
\ofType \TNum, n \ofType \TNum, o \ofType \TNum \to \TNum\}$. But type checking
correctly fails for an annotation $T=\{m \ofType \TNum\}$, because $T \subtpe
U_2$ by transitivity such that $\TNum = U_2.n$ is not satisfiable.

% In the following, we demonstrate how our co-contextual type system works by a concrete example.
% For the example, we assume an extended language with additional numeric types $\TReal$, $\TFloat$
% such that $\TFloat\subtpe\TReal$ and $\TNum\subtpe\TFloat$ hold
% in the subtype relation. This leads to a more interesting typing rule for addition, which
% assigns the most precise numeric type possible, depending on the argument types:
% $$
% \inference[\rulename{T-Add}]
%         {\cojudge {e_1} {T_1} {C_1} {\reqs_1} &
%          \cojudge {e_2} {T_2} {C_2} {\reqs_2} &
%          \ctxcolor{\reqmerge{\reqs_1}{\reqs_2} = \reqs|_{\typectxcolor{C}}} & \fresh{U}}
%         {\cojudge {\add{e_1}{e_2}} {U} {C_1 \cup C_2 \cup \{T_1 \subtpe \TReal, T_2 \subtpe \TReal, \consjoin{U}{T_{1}}{T_{2}}\} \cup \typectxcolor{C}} {\reqs}}.
% $$
% We further consider another primitive operation $\mathit{sqrt}$ with accompanying type rule:
% $$
% \inference[\rulename{T-Sqrt}]
%         {\cojudge {e} {T} {C} {\reqs} }
%         {\cojudge {\mathit{sqrt}(e)} {\TReal} {C \cup \{T \subtpe \TReal\}} {\reqs}}.
% $$
% Now assume we type-check the expression $\mathit{sqrt}(x) + x$ bottom-up, which results in the following constraints
% and requirements:
% \begin{center}
% \begin{tikzpicture}[exp tree]
%   \node (plus) at (1,3) {$+$};
%   \node (sqrt) at (-2.5,0.5) {$\mathit{sqrt}$};
%   \node (sqrt-x) at (-2.5,-1.0) {$x$};
%   \node (x) at (4.5,0.5) {$x$};

%   \annotateNode{sqrt-x}{U_{0}}{\emptyset}{x \ofType U_{0}} 
%   \annotateNode{x}{U_{1}}{\emptyset}{x \ofType U_{1}}
%   \annotateNode{sqrt}{\TReal}{\{U_{0} \subtpe \TReal\}}{x \ofType U_{0}}
%   \annotateNode{plus}{U_{3}}{%
%     \begin{array}{lll}
%       \multicolumn{3}{l}{\vspace{3ex}} \\
%       \{ &U_{0} & \subtpe \TReal, \\
%          & \TReal & \subtpe \TReal, \\
%       & U_{1} & \subtpe \TReal, \\
%       & U_{2} & = U_{0}\wedge U_{1}, \\
%      &  U_{3} & = \TReal \vee U_{1}\}
%     \end{array}%
% }{x \ofType U_{2}}
%   % \node[type node, right=of plus] (plus-type) {\typeNode{\TNumeric}};
%   % \node[ctx node, right=of plus-type] (plus-cons) {\consNode{\{ U_{0} = \TFloat, }};
%   % \node[ctx node, below=of plus-cons] (plus-cons1) {\consNode{U_{1} = \TNumeric, }};
%   % \node[ctx node, below=of plus-cons1] (plus-cons2) {\consNode{ \consmeet{U_{2}}{U_{0}}{U_{1}}  \} }};
%   % \node[ctx node, right=of plus-cons] (plus-req) {\reqNode{x \ofType U_{2}}};

%  \path (plus) edge (x);
%  \path (plus) edge (sqrt);
%  \path (sqrt) edge (sqrt-x);
% \end{tikzpicture}
% \end{center}
% The topmost constraint set is the result of merging the requirements and constraints of the two branches, which both
% yield requirements on variable $x$. From the rule for $\mathit{sqrt}$, we obtain the constraint $U_{0} \subtpe \TFloat$.
% From the refined addition rule, we obtain constraints on the two children of $+$. Both types must be subtypes of $\TReal$
% and the resulting type is their least upper bound $U_{3}$.
% Note the resulting greatest lower bound constraint $U_{2} = U_{0} \wedge U_{1}$, which is in accordance to our
% refined merge operator. Variable $x$ must have at most the types required in the branches to
% ensure consistent usage.
% At this point, we can simplify the constraint set by eliminating the ground constraint $\TReal\subtpe\TReal$ (which
% vacuously holds) to the equivalent set
% \begin{align*}
%   \typecolor{C = } & \typecolor{\{ U_{0}\subtpe\TReal, U_{1}\subtpe \TReal, U_{2} = U_{0}\wedge U_{1}, U_{3} = \TReal \vee U_{1} \}}, 
% \end{align*}
% which we will use throughout the remainder of the example.

% The term is not well-typed, since there are unsatisfied requirements left for $x$.
% Now suppose we close the expression with a lambda abstraction as follows:
% \begin{center}
% \begin{tikzpicture}[exp tree]
%   \node (lambdax) at (1,4) {$\lambda x\ofType \TFloat.$};
%   \node (plus) at (1,3) {$+$};
%   \node (dotdotl) at (0,2.2) {$\ldots$};
%   \node (dotdotr) at (2,2.2) {$\ldots$};

%   \annotateNode{plus}{U_{3}}{\ldots}{x \ofType U_{2}}
%   \annotateNode{lambdax}{\TFloat \to U_{3}}{C\cup\{\TFloat\subtpe U_{2} \}}{\emptyset}
 
%  \path (lambdax) edge (plus);
%  \path (plus) edge (dotdotl);
%  \path (plus) edge (dotdotr);
% \end{tikzpicture}
% \end{center}
% The result is well-typed, since there are no requirements left and 
% the constraint set $C\cup\{\TFloat\subtpe U_{2} \}$ has a solution
% $U_{3} = \TReal$ and $U_{1} = T_{1}$, $U_{2} = T_{2}$, where $T_{1}, T_{2}\in\{\TReal, \TFloat, \TNum\}$
% can be chosen arbitrarily. The type of the entire expression is thus $\TFloat \to\TReal$.


% On the other hand, changing the type of $x$ to the more general $\TNumeric$
% yields the following constraints:
% \begin{center}
% \begin{tikzpicture}[exp tree]
%   \node (lambdax) at (1,4) {$\lambda x\ofType \TNumeric.$};
%   \node (plus) at (1,3) {$+$};
%   \node (dotdotl) at (0,2.2) {$\ldots$};
%   \node (dotdotr) at (2,2.2) {$\ldots$};

%   \annotateNode{plus}{\TNumeric}{\ldots}{x \ofType U_{2}}
%   \annotateNode{lambdax}{\TNumeric \to \TNumeric}{C\cup\{\TNumeric\subtpe U_{2}\}}{\emptyset}
% %=======
% %\TODO{oliver}{I compared to fig~\ref{fig:co-pcf} instead of \ref{fig:pcf-original-constraint}, is this OK?}
% %\todo{mira}{yes}
% %>>>>>>> Stashed changes

 
%  \path (lambdax) edge (plus);
%  \path (plus) edge (dotdotl);
%  \path (plus) edge (dotdotr);
% \end{tikzpicture}
% \end{center}
% The term is ill-typed, since the constraint set is unsatisfiable.
% From the constraints $U_{0} = \TFloat$, $U_{2} = U_{0} \wedge U_{1}$ it must hold that $U_{2} = \TFloat \wedge U_{1}$.
% Together with the definition of the greatest lower bound we obtain $\TFloat \wedge U_{1} \subtpe \TFloat$, i.e.\,
% $U_{2}\subtpe \TFloat$. By the constraint $\TNumeric \subtpe U_{2}$ it follows that $\TNumeric \subtpe \TFloat$, which
% is false.

%%% Local Variables:
%%% mode: latex
%%% TeX-master: "document"
%%% End:
